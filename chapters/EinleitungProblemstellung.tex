\chapter{Einleitung und Problemstellung}

Das Thema dieser Ausarbeitung ist Unsch�rfe Informationen, oder Fuzzy-Sets. Mit Fuzzy-Sets lassen sich schwammige Daten, wie "gro�e" Zahlen oder "mittlere" Temperatur, f�r Maschinen, insbesondere Computern, beschreiben. Diese Mengen k�nnen dann von so genannten Fuzzy-Systeme interpretiert werden. Somit zieht man bestimmte logische R�ckschl�sse bez�glich einer Eingabe. Als Beispiel k�nnte man die Farben von Tomaten nehmen. Der Mensch kann mit h�her Richtigkeit entscheiden, welche Tomate denn reif ist. F�r eine Maschine jedoch ist die Interpretation roher Information(Tomate ist Rot, also reif) nicht m�glich. Mit den Fuzzy-Sets und Regeln kann man in dem System, dieses definieren.

Ziel dieses Projektes ist es, unter Anwendung von Neuronalen Netzen, Parameter beliebiger Fuzzy-Modelle zu optimieren. Das w�rde hei�en, dass bei der Mathematische Funktion $y = a\ast x_1 + b\ast x_2$ die Parametern $a$ und $b$ von dem neuronalen Netz automatisch angepasst werden, sodass sich der Erwartungswert $y'$ bei den Eingaben $x_1$ und $x_2$ ergibt.

In dieser Ausarbeitung werden einige Ans�tze vorgestellt. Schlie�lich wird eine Entscheidung getroffen, welcher Ansatz verwenden wird, Neuro-Fuzzy-Systeme aufzubauen.

%Begonnen werden soll mit einer Einleitung zum Thema, also Hintergrund und Ziel erl�utert werden.

%Weiterhin wird das vorliegende Problem diskutiert: Was ist zu l�sen, warum ist es wichtig, dass man dieses Problem l�st und welche L�sungsans�tze gibt es bereits. Der Bezug auf vorhandene oder eben bisher fehlende L�sungen begr�ndet auch die Intention und Bedeutung dieser Arbeit. Dies k�nnen allgemeine Gesichtspunkte sein: Man liefert einen Beitrag f�r ein generelles Problem oder man hat eine spezielle Systemumgebung oder ein spezielles Produkt (z.B. in einem Unternehmen), woraus sich dieses noch zu l�sende Problem ergibt.
%
%Im weiteren Verlauf wird die Problemstellung konkret dargestellt: Was ist spezifisch zu l�sen? Welche Randbedingungen sind gegeben und was ist die Zielsetzung? Letztere soll das
%beschreiben, was man mit dieser Arbeit (mindestens) erreichen m�chte.