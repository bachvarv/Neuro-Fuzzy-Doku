\chapter{Einleitung und Problemstellung}

Das Thema dieser Ausarbeitung ist es ``Optimierung von Fuzzy-Systeme mit Hilfe von Neuronalen Netzen''. Ein Fuzzy-System ist ein Kontrollsystem, das Analogdaten analysiert, diese durch Fuzzy-Sets in unscharfe Daten umwandelt und folglich eine Inferenz zieht. Mit Fuzzy-Sets lassen sich schwammige Daten, wie ``gro\ss e'' Zahlen oder ``mittlere'' Temperatur, f\"ur Maschinen, insbesondere Computern, beschreiben. Zum Beispiel k\"onnte man die Farben von Tomaten nehmen. Der Mensch kann mit h\"oher Richtigkeit entscheiden, welche Tomate bereits reif ist. F\"ur eine Maschine jedoch ist die Interpretation roher Informationen (Tomate ist rot, also reif) nicht m\"oglich. Mit den Fuzzy-Sets und Regeln k�nnen diese in dem System definiert werden.

Ziel dieses Projektes ist es unter Anwendung von Neuronalen Netzen die Parameter von Fuzzy-Systemen zu optimieren. Es werden sowohl die Fuzzy-Mengen- als auch die Konklusionsparameter angepasst. Daf�r wird der ANFIS-Ansatz eingesetzt.

%Das w\"urde hei�en, dass bei der Mathematische Funktion $y = a\ast x_1 + b\ast x_2$ die Parametern $a$ und $b$ von dem neuronalen Netz automatisch angepasst werden, sodass sich der Erwartungswert $y'$ bei den Eingaben $x_1$ und $x_2$ ergibt.

Zu Beginn der Ausarbeitung werden die Grundlagen f�r Fuzzy-Systeme und Neuronale Netze beschrieben. Weiterhin werden zwei Ans�tze, NEFCON und ANFIS, zur Optimierung von Fuzzy-Systemen vorgestellt. Daraufhin wird der ANFIS-Ansatz im Code gezeigt. Schlie�lich wird das Algorithmus getestet und evaluiert.

%In dieser Ausarbeitung werden einige Ans\"atze vorgestellt. Schlie\ss lich wird eine Entscheidung getroffen, welcher Ansatz verwenden wird, Neuro-Fuzzy-Systeme aufzubauen.

%Begonnen werden soll mit einer Einleitung zum Thema, also Hintergrund und Ziel erl�utert werden.

%Weiterhin wird das vorliegende Problem diskutiert: Was ist zu l�sen, warum ist es wichtig, dass man dieses Problem l�st und welche L�sungsans�tze gibt es bereits. Der Bezug auf vorhandene oder eben bisher fehlende L�sungen begr�ndet auch die Intention und Bedeutung dieser Arbeit. Dies k�nnen allgemeine Gesichtspunkte sein: Man liefert einen Beitrag f�r ein generelles Problem oder man hat eine spezielle Systemumgebung oder ein spezielles Produkt (z.B. in einem Unternehmen), woraus sich dieses noch zu l�sende Problem ergibt.
%
%Im weiteren Verlauf wird die Problemstellung konkret dargestellt: Was ist spezifisch zu l�sen? Welche Randbedingungen sind gegeben und was ist die Zielsetzung? Letztere soll das
%beschreiben, was man mit dieser Arbeit (mindestens) erreichen m�chte.