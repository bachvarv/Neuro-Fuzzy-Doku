\kurzfassung

%% deutsch
\paragraph*{}

Ziel der Ausarbeitung ist es, Fuzzy-Systeme mit Hilfe von Neuronalen Netzen zu optimieren. Es wird ein Programm, eher eine Klasse, erstellt, die ein ANFIS-Modell erzeugt. Das ANFIS-Modell ist ein Neuro-Fuzzy-System, das in seinem Herzen ein TSK-Modell und ein Neuronales Netz kombiniert. Durch das Modell, kann eine spezifische Lernaufgabe mit den Tools des Neuronalen Netzes erlernt werden.  

Vorerst wird aber mit einer kurzen Einf�hrung in die Fuzzy-Logik und Fuzzy-Systeme begonnen. In den folgenden Kapiteln wird auf Neuronale Netze eingegangen. Daraufhin besch�ftige ich mich mit den unterschiedlichen Arten von Neuro-Fuzzy-Systemen. Dabei werden zwei miteinander verglichen - ANFIS und NEFCON. Im letzen Kapitel wird dann auf die Analyse eingegangen. Dieses Kapitel ist die umfangreichste und das wichtigste in dieser Ausarbeitung. Da aber nicht alle Testf�lle betrachtet und analysiert werden k�nnen, wird im hinzugef�gten Anhang zus�tzliche Abbildung zu finden.

%In der Kurzfassung soll in kurzer und pr�gnanter Weise der wesentliche Inhalt der Arbeit beschrieben werden. Dazu z�hlen vor allem eine kurze Aufgabenbeschreibung, der L�sungsansatz sowie die wesentlichen Ergebnisse der Arbeit. Ein h�ufiger Fehler f�r die Kurzfassung ist, dass lediglich die Aufgabenbeschreibung (d.h. das Problem) in Kurzform vorgelegt wird. Die Kurzfassung soll aber die gesamte Arbeit widerspiegeln. Deshalb sind vor allem die erzielten Ergebnisse darzustellen. Die Kurzfassung soll etwa eine halbe bis ganze DIN-A4-Seite umfassen.
%
%Hinweis: Schreiben Sie die Kurzfassung am Ende der Arbeit, denn eventuell ist Ihnen beim Schreiben erst vollends klar geworden, was das Wesentliche der Arbeit ist bzw. welche Schwerpunkte Sie bei der Arbeit gesetzt haben. Andernfalls laufen Sie Gefahr, dass die Kurzfassung nicht zum Rest der Arbeit passt.

%% englisch
%\paragraph*{}
%The same in english.
