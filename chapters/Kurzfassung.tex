\kurzfassung

%% deutsch
\paragraph*{}

In dieser Ausarbeitung wird ein Programm, eher eine Klasse, erstellt, das ein ANFIS-Modell erstellt. Das ANFIS-Modell ist ein Neuro-Fuzzy-System, das in seinem Herzen ein TSK-Modell und ein Neuronales Netz kombiniert werden. Mit dem Modell, darf eine spezifische Lernaufgabe mit den Tools des Neuronalen Netzes gelernt werden. Ziel der Ausarbeitung war das Modell zu implementieren und seine M�glichkeiten zu testen. Im Fazit der letzten Kapitel werden alle Erkenntnisse infolge der Analyse geschrieben.

Vorerst wird aber mit einer kurzen Einf�hrung in Fuzzy-Logik und Fuzzy-Systeme. In den n�chsten Kapitel wird auf Neuronale Netze eingegangen. Folglich besch�ftige ich mich mit den unterschiedlichen Arten von Neuro-Fuzzy-Systeme, dabei werden zwei miteinander verglichen - ANFIS und NEFCON. In dem letzen Kapitel wird dann auf die Analyse eingegangen. Die Kapitel ist die umfangreichste und das wichtigste in dieser Ausarbeitung. Da aber nicht alle Testf�lle betrachtet und analysiert werden k�nnen, wird ein Anhang hinzugef�gt, wo alle zus�tzlichen Testf�llen mit Ergebnissen zu finden sind.

%In der Kurzfassung soll in kurzer und pr�gnanter Weise der wesentliche Inhalt der Arbeit beschrieben werden. Dazu z�hlen vor allem eine kurze Aufgabenbeschreibung, der L�sungsansatz sowie die wesentlichen Ergebnisse der Arbeit. Ein h�ufiger Fehler f�r die Kurzfassung ist, dass lediglich die Aufgabenbeschreibung (d.h. das Problem) in Kurzform vorgelegt wird. Die Kurzfassung soll aber die gesamte Arbeit widerspiegeln. Deshalb sind vor allem die erzielten Ergebnisse darzustellen. Die Kurzfassung soll etwa eine halbe bis ganze DIN-A4-Seite umfassen.
%
%Hinweis: Schreiben Sie die Kurzfassung am Ende der Arbeit, denn eventuell ist Ihnen beim Schreiben erst vollends klar geworden, was das Wesentliche der Arbeit ist bzw. welche Schwerpunkte Sie bei der Arbeit gesetzt haben. Andernfalls laufen Sie Gefahr, dass die Kurzfassung nicht zum Rest der Arbeit passt.

%% englisch
%\paragraph*{}
%The same in english.
